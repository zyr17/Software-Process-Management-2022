
\chapter{引言}\label{chap:introduction}

随着软件规模越来越大,同时开发效率和开发周期要求变高,传统的软件开发、测试、部署流程已经难以适应当前的开发需求。
如果仍然依靠手动编译、测试和部署,将会大大降低开发的效率。例如一个包含前端、后端、数据库等模块的系统,
当其中一个模块修改后,需要测试各个不同模块都能够正常工作,需要编译部署所有相关模块并执行各自的测试用例,
在通过测试后将模块上线部署。这一系列工作涉及不同环境,步骤复杂,因此很有必要引入自动化,
在代码进行提交和合并的时候自动进行,从而提升代码部署效率,同时能够更快的发现和定位问题,减少上线事故。

在软件开发实践中,对代码自动编译、测试和发布称为持续集成,而将通过集成的代码自动上线部署称为持续部署。
下面两节会对这两个方向的概念和涉及的部分技术细节进行介绍。

\section{持续集成}

持续集成(Continuous Integration, CI)是一种软件开发实践,当开发人员将代码合并至主分支时,持续集成系统会自动
根据最新合并代码,对项目进行构建并进行测试。使用持续集成,可以避免大家联合开发时,由于代码冲突或者对代码的
非预期修改导致代码合并后功能出错。同时,也可以让开发人员从繁重复杂的代码编译和测试中解放出来,
提升代码开发效率和稳定性。

使用持续集成,需要在进行软件开发时满足一定的条件。一般来说具有下述要求:

\subsection{使用代码版本控制系统}

在协同开发中,为了能够方便的进行代码合并和同步,对代码进行版本控制几乎是必不可少的。目前最流行的版本控制工具
是Git。在持续集成中,版本控制工具也是必须工具之一。对代码进行版本控制,可以选择进行持续集成的条件,
如仅对主分支的合并和提交进行持续集成,或是仅对代码标签进行持续集成,这样避免了对开发中的代码进行不必要的持续集成操作,
从而减少持续集成的资源消耗,同时也可以减少无意义的持续集成错误提示。

\subsection{完善的单元测试用例}

为了最大限度的发挥持续集成的优势,完善的单元测试是必不可少的。通过单元测试,持续集成时可以确认对代码的修改与之前
实现的功能没有冲突。同时,通过检查单元测试下代码的覆盖率,也可以发现单元测试的漏洞,提升测试有效性。
虽然仅靠单元测试不能发现所有代码非预期执行的问题,但是已经可以高效找到大部分显著的问题,避免了代码上线后
在生产环境出错所造成的的影响。

\subsection{持续集成服务器}
显然,由于持续集成时需要编译构建项目和运行测试代码,一台用于持续集成的服务器是必不可少的。
目前很多代码托管平台和云平台提供了持续集成的服务,不少平台的服务面向开源代码或是小规模团体是免费的。
服务器会持续监控被托管的代码,当代码满足集成触发要求时自动进行项目构建、测试和产物打包。

\section{持续部署}

持续部署(Continuous Deployment, CD)是在代码通过持续集成后将代码自动部署至生产环境,减少人工成本,提升部署效率的过程。
将代码部署到生产环境需要不小的工作量,尤其是一些大型分布式软件、或者是涉及多种运行环境不同的组件,
人工部署步骤繁杂,耗时且容易出现失误。通过持续部署,开发人员可以自动将新开发的软件部署到生产环境,
同时还可以支持特定的部署规则,例如灰度部署、自动回滚等,可以显著减轻运维的成本。

由于容器技术具有环境统一互不干扰、占用资源小、体积小、迁移扩展方便等显著优点,使用容器部署成为了目前线上任务的
主流方式之一。正由于容器有上述优点,在使用持续部署的过程中使用容器也是最合适的方式之一。同时,
Kubernetes(k8s)作为新兴的大规模容器编排技术,在大规模管理和部署容器上具有明显优势,
成为目前业界的主流。

由于持续部署一般涉及大型商用软件,同时普通用户一般没有很强的持续部署需求,所以相比持续集成,
提供持续部署服务的不是很多,同时功能比较简单。简单的持续部署功能在大部分开发套件中能够找到,
可以实现简单的持续部署。

\section{持续集成和持续部署平台}

目前有很多提供持续集成和持续部署的平台,它们各有特色。本文会针对接触过的三个持续集成部署平台AppVeyor、GitHub Actions和
华为云进行介绍。同时,基于本次课程实践,本文对本次课程实践中代码编写,持续集成和持续部署部分进行介绍,
也对其中遇到的困难和体会进行总结。